\chapter{Présentation du contexte}

Dans cette partie, nous présentons le contexte du projet de l'UV MGP321 ainsi que le contexte de notre projet en lui même.

\section{L'UV MGP321}

Le module projet \gls{MGP}321 est proposé aux élèves {FIP}3A. Pendant 63 heures, chaque binôme d'élèves réalise un projet de développement technique. Ce module permet aux élèves une mise en application pratique d'éléments de cours vus pendant le cursus de formation. C'est de plus une occasion pour les élèves de se confronter à des technologies différentes de leur activité d'entreprise. Les élèves sont libres de choisir dans la liste des projets Télécoms et des projets Info-Réseau. Les élèves d'un binôme peuvent donc être d'options différentes. L'encadrement d'un projet peut être fait par plusieurs enseignants.\\ 

Les objectifs pédagogiques sont multiples, aussi bien techniques que méthodologiques : \\

\begin{itemize}
    \item Exploiter ses connaissances, en les élargissant au besoin, pour répondre à un besoin client

\item Reformuler ce besoin sous forme de cahier des charges fonctionnel
\item Établir et gérer le déroulement du travail en équipe, être réactif, savoir s'adapter aux aléas
\item Utiliser efficacement des outils de travail collaboratif (forge logicielle, outils de planification, ...)
\item Rédiger un rapport de projet qui répond aux attentes des différentes parties
\item Faire une présentation synthétique et critique de ses travaux \\

\end{itemize}

\par
\\ Les livrables suivants concluent le projet : \\
\begin{itemize}
    

\item une réalisation technique avec analyse critique des résultats présentée devant un jury
\item un rapport écrit
\item une soutenance orale
\end{itemize}

\section{Le projet Sedela}

Le projet Sedela "Self Data for Enhancing Lifelong learning Autonomy" est un projet de recherche qui travaille sur le contrôle d'usage partagé de donnée. C'est dans ce contexte que s'inscrit notre projet de MGP321. \\

Dans ce cadre, une application web doit être expérimentée et être testée par des vrais étudiants dans quelques mois. Cet outil prendra la forme d'un portfolio et devra répondre à plusieurs requêtes. La liste des fonctions qui sont attendues à terme sont disponibles dans la partie cahier des charges de ce rapport. Le but du projet est de commencer le développement de cet outil. \\

Un prototype de portfolio avait déjà été réalisé l'année passée par un étudiant de Télécom Bretagne. L'idée initiale pour notre projet de MGP321 était de s'inspirer, voire de repartir de ce prototype pour le compléter pour se mettre en accord avec les nouveaux besoins du client. Ensuite, notre projet sera poursuivi par un autre étudiant de Télécom Bretagne. Il est donc très important de faire en sorte que le projet soit pérenne, notamment en terme d'architecture ou de technologie utilisée afin de faciliter la poursuite de son développement. La première étape de ce projet a donc été de faire une analyse de l'existant, afin de mieux appréhender une éventuelle reprise de celui-ci.





