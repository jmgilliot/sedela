\chapter{Bilan \& résultats}

Dans ce chapitre, nous allons faire un bilan d'un point de vue technique et organisationnel.

\section{Bilan technique}

\subsection{Avancement du projet}

Concernant l'avancement du projet, nous sommes très déçus de ne pas avoir pu réaliser toutes les fonctions initialement prévues. Nous n'avons souvent pas réussis à trouver des réponses à nos questions techniques par le manque de documentation présentes sur le web. De plus, comme c'est une architecture et des façons de développer différentes (client-side app), nous avons mis beaucoup de temps à tenter de comprendre comment développer telle ou telle fonction. \\

Peut être que nous avons été trop ambitieux sur le nombre de fonction à réaliser, mais nous n'aurions pas pu prévoir le changement de programme (nous devions repartir de zéro) qui est arrivé lors de la deuxième semaine de notre projet.

\subsection{Apprentissage technique}

D'un point de vue technique, nous avons pu acquérir de nombreuses connaissances. En effet, avant le projet, seul Jordan connaissait le NodeJS et AngularJs. Maintenant, nous sommes à l'aise sur ces technologies. De plus, nous avons appris le concept de création d'une app client-side, ce qui est intéressant pour nous pour la suite de notre cursus.

\section{Bilan organisationnel du projet}

\subsection{Répartition des tâches}

Au cours de ce projet, nous avons beaucoup travaillé ensemble à la compréhension de l'architecture de l'application réalisée par Cédric Patchane l'an dernier. Cela nous a permis de nous expliquer mutuellement les choses que l'on comprenaient et celles que l'on ne comprenaient pas. \\

Puis, quand nous avons commencé à développer, comme nous avons fait face à de gros soucis technique avec Cozy, nous nous sommes beaucoup aidés mutuellement. 

Cependant, lors de la réalisation de la partie graphique, par exemple, nous nous sommes séparés les tâches : chacun à réalisé plusieurs des pages proposées.



